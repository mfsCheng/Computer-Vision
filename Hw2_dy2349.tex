\documentclass[letterpaper,flegn,12pt]{extarticle}
\usepackage{tabularx} % extra features for tabular environment
\usepackage{amsmath}  % improve math presentation
\usepackage{graphicx} % takes care of graphic including machinery
\usepackage[margin=1in,letterpaper]{geometry} % decreases margins
\usepackage{cite} % takes care of citations
\usepackage[final]{hyperref} % adds hyper links inside the generated pdf file
\usepackage{enumerate}
\usepackage{float}
\usepackage{graphicx}
\hypersetup{
	colorlinks=true,       % false: boxed links; true: colored links
	linkcolor=blue,        % color of internal links
	citecolor=blue,        % color of links to bibliography
	filecolor=magenta,     % color of file links
	urlcolor=blue         
}
%++++++++++++++++++++++++++++++++++++++++


\begin{document}

\title{COMSW 4731 Computer Vision, Fall 2017
\\Homework 2}
\author{Danwen Yang - dy2349@columbia.edu}
\date{\today}
\maketitle

\section*{Problem 1}
Show that the extreme values of moment of inertia$(E)$ for a 2D binary object are given by the eigenvalues of the 2 x 2 matrix:
\begin{equation}
\nonumber
\begin{bmatrix}
	a & b/2 \\
	b/2 & c
\end{bmatrix}
\end{equation}
\textbf{1.} where $a, b, c,$ and $E$ are as defined in the lecture notes. \textbf{(3 points)}
\\ \textbf{Proof}:
\\ The characteristic equation of the matrix above is
\begin{equation}
\nonumber
\begin{vmatrix}
	\textbf{A} - \lambda \cdot \textbf{I}
\end{vmatrix}
= 
\begin{vmatrix}
	\begin{bmatrix}
		a & b/2 \\
		b/2 & c
	\end{bmatrix}
	-
	\begin{bmatrix}
		\lambda & 0 \\
		0 & \lambda
	\end{bmatrix}
\end{vmatrix}
= 0
\end{equation}
\begin{equation}
\nonumber
	\begin{vmatrix}
		a-\lambda & b/2 \\
		b/2 & c-\lambda
	\end{vmatrix}
	= (a-\lambda)(c-\lambda)-(b/2)^2 = 0
\end{equation}
\begin{equation}
\nonumber \lambda^2 -(a+c)\lambda+ac-\frac{b^2}{4} = 0
\end{equation}
and the two eigenvalues are 
\begin{equation}
\nonumber
\begin{aligned}
	\lambda_{1,2} &= \frac{a+c \pm \sqrt{(a+c)^2-4ac+b^2}}{2} \\
	&= \frac{a+c \pm \sqrt{(a-c)^2+b^2}}{2}
\end{aligned}
\end{equation}
By the definition of $a, b, c,$ and $E$ in the lecture notes.
\begin{equation}
\nonumber
\begin{aligned}
	E &= a sin^2\theta - b sin\theta cos\theta + c cos^2\theta \\
	E &= \frac{1}{2}(a+c) - \frac{1}{2}(a-c)cos2\theta - \frac{1}{2}sin2\theta
\end{aligned}
\end{equation}
When $tan2\theta = \dfrac{b}{a-c}$, that is to say $sin2\theta = \pm \dfrac{b}{\sqrt{b^2 + (a-c)^2}}$ and $cos2\theta = \pm \dfrac{a-c}{\sqrt{b^2 + (a-c)^2}}$, $E$ has extreme values. \\
When $sin2\theta = \dfrac{b}{\sqrt{b^2 + (a-c)^2}}$ and $cos2\theta = \dfrac{a-c}{\sqrt{b^2 + (a-c)^2}}$, $E$ has the minimum value:
\begin{equation}
\begin{aligned}
\nonumber E &= \frac{1}{2}(a+c) - \frac{(a-c)^2}{2\sqrt{b^2+(a-c)^2}} - \frac{b^2}{2\sqrt{b^2+(a-c)^2}} \\
E &= \frac{1}{2}(a+c) - \frac{\sqrt{b^2+(a-c)^2}}{2}
\end{aligned}
\end{equation}
When $sin2\theta = -\dfrac{b}{\sqrt{b^2 + (a-c)^2}}$ and $cos2\theta = -\dfrac{a-c}{\sqrt{b^2 + (a-c)^2}}$, $E$ has the maximum value:
\begin{equation}
\begin{aligned}
\nonumber E &= \frac{1}{2}(a+c) + \frac{(a-c)^2}{2\sqrt{b^2+(a-c)^2}} + \frac{b^2}{2\sqrt{b^2+(a-c)^2}} \\
E &= \frac{1}{2}(a+c) + \frac{\sqrt{b^2+(a-c)^2}}{2}
\end{aligned}
\end{equation}
As we can see above, the extreme values of moment of inertia$(E)$ for a 2D binary object are given by the eigenvalues of the matrix.
~\\
\\\textbf{2.} Argue that $E$ is real and non-negative, and hence prove that $4ac \geq b^2$. \textbf{(2 points)}
\\\textbf{Proof}:
\\By the definition of moment of inertia $E$ for a 2D binary object
\begin{equation}
\nonumber E = \int\int_I r^2b(x, y)dxdy
\end{equation}
Where $r$ is the perpendicular distance from point $(x, y)$ to a line. Obviously, $r^2$ is real and nonnegative.
\\ Since the area of a object is 
\begin{equation}
\nonumber A = \int\int_I b(x, y)dxdy
\end{equation}
which is real and nonnegative.
So $E$ is real and nonnegative.
\\ And the minimum value of $E$ is nonnegative
\begin{equation}
\begin{aligned}
\nonumber \frac{1}{2}(a+c) - \frac{\sqrt{b^2+(a-c)^2}}{2} &\geq 0 \\
(a+c)^2 &\geq b^2 + (a-c)^2 \\
4ac &\geq b^2 
\end{aligned}
\end{equation}
~\\
\\\textbf{3.} What kind of object gives $E$ equal to zero ? \textbf{(1 point)}
\\ \textbf{Solution}:
\\When the binary object is a straight line, $E$ is zero.
\\Because $E$ is the integer of the square of the distance from the point $(x, y)$ to a line, if $E$ is zero, then every distance is zero, which means every point is on the line, and the binary object is a straight line.

\end{document}